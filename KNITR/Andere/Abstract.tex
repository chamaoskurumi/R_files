
\selectlanguage{ngerman} 
\begin{center}
    {\normalsize{\bf Masterarbeit}} \\\vspace{0.5cm}
     {\Large{\bf Aufwertung und Verdrängung in Berlin: \\
                Räumliche Analysen zur Messung von Gentrifizierung}} \vspace{1.0cm} \\
    {\bf Guido Schulz} \\ \vspace{2cm}

\end{center}
\selectlanguage{english} 
\begin{abstract}
conjunction of social and real estate valorisation. displacement
\end{abstract}
\selectlanguage{ngerman} 
\begin{abstract}
Ziel dieser Arbeit war es sowohl Gentrifizierungsgebiete in Berlin auf Nachbarschaftsniveau zu identifizieren und vergleichend zu charakterisieren, als auch den für Gentrifizierungsprozesse wesentlichen Zusammenhang zwischen Aufwertung und Verdrängung zu messen werden. 
Mit der kleinräumigen, systematischen Erfassung von Aufwertung und Verdrängung versuchte diese Arbeit einerseits zur Schließung einer Lücke in der deutschen Gentrifizierungsliteratur beizutragen, sowie andererseits methodische Unzulänglichkeiten vergleichbarer internationaler Studien zu beheben. Mithilfe der teils weiterentwickelten, teils neu ausgearbeiteten Methodik konnten robuste Belege für den Zusammenhang zwischen immobilienwirtschaftlicher und sozialer Aufwertung, sowie für die Verknüpfung von Aufwertung und Verdrängung gesammelt werden. Im Wesentlichen entsprechen alle empirischen Ergebnisse dieser Arbeit den gängigen Theoremen der Gentrifizierungsforschung - und dies obwohl der informelle Wohnungsmarkt mitsamt der Dunkelziffer behördlich nicht erfasster Armut und wohnräumlicher Mobilität die Analyse vermutlich in unbekanntem Ausmaß verzerrt hat. Dass trotz dieser Verzerrungen, der unterschiedlichen Erhebungsmethoden und Kontexte die für Berlin geschätzten ökonomischen Verdrängungsraten in etwa den von \citet{Newman2006a} berechneten Raten für New York City entsprechen, ist bemerkenswert.
\end{abstract}


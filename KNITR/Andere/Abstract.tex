
\selectlanguage{ngerman} 
\begin{center}
     {\Large{\bf Aufwertung und Verdrängung in Berlin: \\
                Räumliche Analysen zur Messung von Gentrifizierung}} \vspace{1.0cm} \\
    {\bf Guido Schulz} \\ \vspace{2cm}

\begin{abstract}
Trotz des boomenden Berliner Immobilienmarktes, der großen medialen Aufmerksamkeit für das Thema der Gentrifizierung und der von stadtpolitischen Protestbewegungen kritisierten sozialen Auswirkungen von Aufwertungsprozessen, wurden in Berlin bisher kaum Studien zur Messung von Gentrifizierung unternommen. Mit dem Versuch einer stadtweit kleinräumigen Erfassung von Aufwertung und Verdrängung soll mit dieser Arbeit sowohl zur Schließung der Lücke in der deutschen Literatur zu Gentrifizierung, als auch zur Behebung methodischer Mängel vergleichbarer internationaler Studien beigetragen werden. Durch die Analyse von Aggregatsdaten des Beobachtungszeitraums von 2007-2012 konnten mithilfe eigens (weiter)entwickelter deskriptiver wie inferentieller statistischer Methoden Gentrifizierungsgebiete auf Nachbarschaftsniveau identifiziert und vergleichend charakterisiert werden. Überdies wurde ein starker Zusammenhang zwischen Aufwertung und Verdrängung nachgewiesen. So konnte gezeigt werden, dass Gentrifizierungsgebiete im Vergleich zu entsprechenden Kontrollgebieten viel höhere Binnenfortzugs- und Außenzuzugsraten besaßen. Im Zusammenhang mit den dort vorherrschenden intensiven sozialen und immobilienwirtschaftlichen Aufwertungsdynamiken ließ dies auf eine Verdrängung ärmerer Bevölkerungsgruppen schließen. Zudem ergab eine durch gewichtete, \textit{Spatial Autoregressive Models} (SAR) ermittelte Schätzung ökonomischer Verdrängungsraten ähnliche Werte, wie sie von Newman und Wyly im Jahre 2006 für New York City errechnet wurden.
\end{abstract}

\newpage
\selectlanguage{english} 
{\Large{\bf Valorisation and Displacement in Berlin: \\
                Spatial Analyses for Measuring Gentrification}} \vspace{1.0cm} \\
     {\bf Guido Schulz} \\ \vspace{2cm}

\begin{abstract}
The real-estate market in Berlin is booming, the issue of gentrification has recently received increased media interest and urban protest movements are condemning the social consequences of real-estate valorisation. Despite all this attention, research on measuring gentrification in Berlin has been scarce. By attempting to capture real-estate valorisation and its associated social upgrading and displacement processes, this study sought to close a gap in the German gentrification literature and also aimed to resolve methodological shortcomings of comparable international studies. By refining and developing descriptive as well as inferential statistical methods applied to aggregate data from 2007-2012, neighbourhoods experiencing gentrification could be identified and characterized. Furthermore, a strong relationship between real-estate valorisation and displacement could be verified. A comparison between gentrification and control areas revealed that the gentrification areas exhibit much higher mobility rates. Given the intense dynamics of social upgrading and real-estate valorisation in these areas, the high mobility rates strongly suggested that financially disadvantaged residents were being displaced. In addition, an estimation of economic displacement rates via weighted \textit{Spatial Autoregressive Models} (SAR) yielded similar values to those Newman and Wyly had calculated for New York City back in 2006.
\end{abstract}
\end{center} 
\selectlanguage{ngerman}

\section{Einleitung}

\begin{itemize}

    \item What is the subject of the study? Describe the
        economic/econometric problem.
    \item What is the purpose of the study (working hypothesis)?
    \item What do we already know about the subject (literature
        review)? Use citations: shows that...
        Alternative Forms of the Wald test are considered
    \item What is the innovation of the study?
    \item Provide an overview of your results.
    
"`I would challenge [...] anyone to show how census data can provide convincing evidence of displacement. It can show that one class or housing tenure has grown and that another has declined but it cannot show displacement."' Hamnet 2010 S.~184

Gentrimap beinhaltet keine Analyse von zwischen Aufwertungsprozessen und Wanderungsverhalten.

Versuch Verdrängung in Berlin im Zweitraum zwischen 2007-2012 statistisch kleinräumlig zu erfassen und 


"`it is essential to have conceptual clarity before research on displacement begins, and before any conclusions can be drawn"' \citet[S.~304]{Slater2009}

Eine Analyse (von altgriech. ἀνάλυσις analysis „Auflösung“) ist eine systematische Untersuchung, bei der das untersuchte Objekt oder Subjekt in Bestandteile (Elemente) zerlegt und auf Grundlage von Kriterien erfasst werden. Anschließend werden diese geordnet, untersucht und ausgewertet. Insbesondere betrachtet man Beziehungen und Wirkungen (oft: Wechselwirkungen) zwischen den Elementen.

"`Eine Analyse (von altgriech analysis 'Auflösung') ist eine systematische Untersuchung, bei der das untersuchte Objekt oder Subjekt in Bestandteile (Elemente) zerlegt und auf Grundlage von Kriterien erfasst werden. Anschließend werden diese geordnet, untersucht und ausgewertet. Insbesondere betrachtet man Beziehungen und Wirkungen (oft: Wechselwirkungen) zwischen den Elementen."' Wikipedia


Keine qualitativen Methoden, 

Durch die statistische Modellierung des Zusammenhang zwischen immoblienwirtschaftlichen sowie sozialen Aufwertungsprozessen und wohnräumlicher Mobilität sollen empirische Erkenntnisse bezüglich der Existenz, Lokalisierung und Quantifizierung von Gentrification und Verdrängung in Berlin gewonnen werden. \\
Zunächst soll mit Hilfe stadtsoziologischer Theorie der konzeptionelle Rahmen für die Studie bestimmt und Begriffsdefinitionen gesetzt werden. Auf Grundlage des aktuellen Forschungsstandes werden Forschungsfragen entwickelt und Hypothesen formuliert. \\
Im ersten Schritt sollen mit den Methoden explorativer Datenanalyse eine räumliche Identifizierung und Quantifizierung von Gentrificationprozessen in Berlin vorgenommen werden. Die identifizierten Aufwertungsgebiete werden in einem Vergleich mit entsprechenden Kontrollgebieten genauer charaktierisiert. In einem zweiten Schritt werden räumlich autoregressive Modelle \textit{(\ac{SAR})} geschätzt um einerseits den Effekt immobilienwirtschaftlicher Aufwertungsdynamik auf die wohnräumliche Mobilität der Berliner Bevölkerung zu messen und andereseits um den Charakter und Ausmaß der Interaktion zwischen sozialer Aufwertungsdynamik und wohnräumlicher Mobilität zu bestimmen. Die Ergebnisse der Regressionsmodelle werden dann im Kontext Marcuses Verdrängungstypen intepretiert und analysiert.
\end{itemize}

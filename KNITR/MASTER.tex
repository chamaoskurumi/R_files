\documentclass[a4paper,11pt]{article}\usepackage[]{graphicx}\usepackage[]{color}
%% maxwidth is the original width if it is less than linewidth
%% otherwise use linewidth (to make sure the graphics do not exceed the margin)
\makeatletter
\def\maxwidth{ %
  \ifdim\Gin@nat@width>\linewidth
    \linewidth
  \else
    \Gin@nat@width
  \fi
}
\makeatother

\definecolor{fgcolor}{rgb}{0.345, 0.345, 0.345}
\newcommand{\hlnum}[1]{\textcolor[rgb]{0.686,0.059,0.569}{#1}}%
\newcommand{\hlstr}[1]{\textcolor[rgb]{0.192,0.494,0.8}{#1}}%
\newcommand{\hlcom}[1]{\textcolor[rgb]{0.678,0.584,0.686}{\textit{#1}}}%
\newcommand{\hlopt}[1]{\textcolor[rgb]{0,0,0}{#1}}%
\newcommand{\hlstd}[1]{\textcolor[rgb]{0.345,0.345,0.345}{#1}}%
\newcommand{\hlkwa}[1]{\textcolor[rgb]{0.161,0.373,0.58}{\textbf{#1}}}%
\newcommand{\hlkwb}[1]{\textcolor[rgb]{0.69,0.353,0.396}{#1}}%
\newcommand{\hlkwc}[1]{\textcolor[rgb]{0.333,0.667,0.333}{#1}}%
\newcommand{\hlkwd}[1]{\textcolor[rgb]{0.737,0.353,0.396}{\textbf{#1}}}%

\usepackage{framed}
\makeatletter
\newenvironment{kframe}{%
 \def\at@end@of@kframe{}%
 \ifinner\ifhmode%
  \def\at@end@of@kframe{\end{minipage}}%
  \begin{minipage}{\columnwidth}%
 \fi\fi%
 \def\FrameCommand##1{\hskip\@totalleftmargin \hskip-\fboxsep
 \colorbox{shadecolor}{##1}\hskip-\fboxsep
     % There is no \\@totalrightmargin, so:
     \hskip-\linewidth \hskip-\@totalleftmargin \hskip\columnwidth}%
 \MakeFramed {\advance\hsize-\width
   \@totalleftmargin\z@ \linewidth\hsize
   \@setminipage}}%
 {\par\unskip\endMakeFramed%
 \at@end@of@kframe}
\makeatother

\definecolor{shadecolor}{rgb}{.97, .97, .97}
\definecolor{messagecolor}{rgb}{0, 0, 0}
\definecolor{warningcolor}{rgb}{1, 0, 1}
\definecolor{errorcolor}{rgb}{1, 0, 0}
\newenvironment{knitrout}{}{} % an empty environment to be redefined in TeX

\usepackage{alltt}

\usepackage{amsmath,amssymb,amsfonts,amsthm}    % Typical maths resource packages
\usepackage{graphicx}                           % Packages to allow inclusion of graphics
\usepackage{hyperref}                           % For creating hyperlinks in cross references
\usepackage[ngerman]{babel}
\usepackage[backend=bibtex, style=authoryear, language=ngerman, isbn=false, url=false, doi=false, eprint=false, natbib=true, citestyle=authoryear]{biblatex}
\usepackage[utf8]{inputenc}
\usepackage{hyperref}

\bibliography{Bibliographie}
% -------------------------------
% --- some layout definitions ---
% -------------------------------

% define topline
\usepackage[automark]{scrpage2}
\pagestyle{scrheadings}
\automark{section}
\clearscrheadings
\ohead{\headmark}

% define page size, margin size
\setlength{\headheight}{2\baselineskip}
\voffset=-2cm
\hoffset=-3cm
\textheight24cm
\textwidth15.5cm
\topmargin1cm
\oddsidemargin3cm
\evensidemargin3cm

\renewcommand{\baselinestretch}{1.5}
\renewcommand{\labelitemii}{-}
\IfFileExists{upquote.sty}{\usepackage{upquote}}{}
\begin{document}










% -------------------------------
% --- frontmatter: Title page ---
% -------------------------------

\thispagestyle{empty}
\begin{center}
    {\normalsize{\bf Masterarbeit}} \\\vspace{0.5cm}
    {\LARGE{\bf Räumlich Autoregressive Modelle zur Messung von Verdrängung in Berlin}} \vspace{0.5cm}

    {\normalsize eingereicht beim Betreuer}\\\vspace{0.5cm}
    {\normalsize{\bf Prof. Dr. Axel Werwartz}} \\\vspace{0.5cm}
    {\normalsize Technische Universität Berlin \\
    Fakultät VII - Wirtschaft und Management \\
    Institut für Volkswirtschaftslehre und Wirtschaftsrecht \\
    FG Ökonometrie und Wirtschaftsstatistik} \vspace{1cm}

    {\normalsize sowie beim Zweitgutachter}\\\vspace{0.5cm}
    {\normalsize{\bf Dr. Andrej Holm}} \\\vspace{0.5cm}
    {\normalsize Humboldt-Universität zu Berlin \\
    Kultur-, Sozial- und Bildungswissenschaftliche Fakultät\\
    Institut für Sozialwissenschaften\\
    Stadt- und Regionalsoziologie} \vspace{1cm}

    {\normalsize vorgelegt von \\\vspace{0.5cm}
    {\bf Guido Schulz} \\
    Matrikelnummer: HU 554142 \\
    \href{mailto:schulzgu@student.hu-berlin.de}{schulzgu@student.hu-berlin.de}} \vspace{1cm}


    {\normalsize Zur Erlangung des akademischen Grades eines \\
    {\bf Master of Science (M.Sc.) in Statistik} \\
    Berlin, der 31. Juni 2015}

\end{center}


% ------------------------------------
% --- frontmatter: Acknowledgement ---
% ------------------------------------
\newpage
\pagestyle{plain}
\pagenumbering{roman}   % define page number in roman style
\setcounter{page}{1}    % start page numbering
\section*{Danksagung}

I would like to thank




% -----------------------------
% --- frontmatter: Abstract ---
% -----------------------------
\newpage

\selectlanguage{ngerman} 
\begin{center}
    {\normalsize{\bf Masterarbeit}} \\\vspace{0.5cm}
     {\Large{\bf Aufwertung und Verdrängung in Berlin: \\
                Räumliche Analysen zur Messung von Gentrifizierung}} \vspace{1.0cm} \\
    {\bf Guido Schulz} \\ \vspace{2cm}

\end{center}
\selectlanguage{english} 
\begin{abstract}
conjunction of social and real estate valorisation. displacement
\end{abstract}
\selectlanguage{ngerman} 
\begin{abstract}
Ziel dieser Arbeit war es sowohl Gentrifizierungsgebiete in Berlin auf Nachbarschaftsniveau zu identifizieren und vergleichend zu charakterisieren, als auch den für Gentrifizierungsprozesse wesentlichen Zusammenhang zwischen Aufwertung und Verdrängung zu messen werden. 
\end{abstract}



% -----------------------------
% --- frontmatter: Contents ---
% -----------------------------
\newpage
\tableofcontents
\clearpage



% ----------------------------------------------------
% --- frontmatter: List of Figures (not mandatory) ---
% ----------------------------------------------------
\newpage
\addcontentsline{toc}{section}{Abkürzungsverzeichnis}
\ohead[]{Abkürzungsverzeichnis}
\section*{Abkürzungsverzeichnis}

\begin{acronym}
 \acro{SAR}{Simultaneous autoregressive model}
 \acro{CAR}{Conditional autoregressive model}
 \acro{LOR}{Lebensweltlich orientierte Räume}
 \acro{AIC}{Akaike information criterion}
\end{acronym}




% ----------------------------------------------------
% --- frontmatter: List of Figures (not mandatory) ---
% ----------------------------------------------------
\newpage
\addcontentsline{toc}{section}{Abbildungsverzeichnis}
\ohead[]{\rightmark}
\listoffigures



% ---------------------------------------------------
% --- frontmatter: List of Tables (not mandatory) ---
% ---------------------------------------------------
\newpage
\addcontentsline{toc}{section}{Tabellenverzeichnis}
\listoftables



% -------------------------------
% --- main body of the thesis ---
% -------------------------------
\newpage
\pagestyle{plain}
\setcounter{page}{1}    % start page numbering anew
\pagenumbering{arabic}  % page numbers in arabic style


\section{Einleitung}

\begin{itemize}

    \item What is the subject of the study? Describe the
        economic/econometric problem.

    \item What is the purpose of the study (working hypothesis)?

    \item What do we already know about the subject (literature
        review)? Use citations: shows that...
        Alternative Forms of the Wald test are considered

    \item What is the innovation of the study?

    \item Provide an overview of your results.


    \item The introduction should not be longer than 4 pages.
    

\end{itemize}






\section{Method/Model/Theory}\label{Sec:Method}

\begin{itemize}

    \item How was the data analyzed ?

    \item Present the underlying economic model/theory and
        give reasons why it is suitable to answer the given problem.

    \item Present econometric/statistical estimation method and
        give reasons why it is suitable to answer the given problem.

    \item Allows the reader to judge the validity of the study and
        its findings.

    \item Depending on the topic this section can also be split up
        into separate sections.

\begin{knitrout}
\definecolor{shadecolor}{rgb}{0.969, 0.969, 0.969}\color{fgcolor}
\includegraphics[width=\maxwidth]{figure/unnamed-chunk-3-1} 

\end{knitrout}


\end{itemize}

\newpage





\section{Data}\label{Sec:Data}

\begin{itemize}

    \item Describe the data and its quality.
    \item How was the data sample selected?
    \item Provide descriptive statistics such as:
        \begin{itemize}
            \item time period,
            \item number of observations, data frequency,
            \item mean, median,
            \item min, max, standard deviation,
            \item skewness, kurtosis, Jarque--Bera statistic,
            \item time series plots, histogram.
        \end{itemize}
    \item For example:
        \begin{table}[ht]
        \begin{center}
            {\footnotesize
            \begin{tabular}{l|cccccccccc}
                \hline \hline
                           & 3m    & 6m    & 1yr   & 2yr   & 3yr   & 5yr   & 7yr   & 10yr  & 12yr  & 15yr   \\
                \hline
                    Mean   & 3.138 & 3.191 & 3.307 & 3.544 & 3.756 & 4.093 & 4.354 & 4.621 & 4.741 & 4.878  \\
                    StD    & 0.915 & 0.919 & 0.935 & 0.910 & 0.876 & 0.825 & 0.803 & 0.776 & 0.768 & 0.762  \\
                \hline \hline
            \end{tabular}}
        \end{center}
        \caption{Some descriptive statistics of location and dispersion for
        2100 observed swap rates for the period from February 15, 1999
        to March 2, 2007. Swap rates measured as 3.12 (instead of 0.0312). See Table
        \ref{Tab:DescripStatsRawDataDetail} in the appendix for
        more details.}
        \label{Tab:DescripStatsRawData}
        \end{table}

    \item Allows the reader to judge whether the sample is biased or to evaluate possible impacts of outliers, for
    example.
    
\begin{knitrout}
\definecolor{shadecolor}{rgb}{0.969, 0.969, 0.969}\color{fgcolor}\begin{kframe}
\begin{verbatim}
##       PLZ            Zeit         Miete_H1         Miete_H2     
##  10115  :  11   Min.   :2004   Min.   : 3.700   Min.   : 3.800  
##  10117  :  11   1st Qu.:2006   1st Qu.: 5.500   1st Qu.: 5.500  
##  10119  :  11   Median :2009   Median : 6.300   Median : 6.400  
##  10178  :  11   Mean   :2009   Mean   : 6.586   Mean   : 6.712  
##  10179  :  11   3rd Qu.:2012   3rd Qu.: 7.300   3rd Qu.: 7.500  
##  10243  :  11   Max.   :2014   Max.   :14.000   Max.   :14.000  
##  (Other):2024                  NA's   :34       NA's   :27
\end{verbatim}
\end{kframe}
\includegraphics[width=\maxwidth]{figure/unnamed-chunk-4-1} 

\end{knitrout}

\cite{Wyly2010} is auchs schön

\end{itemize}

\newpage





\section{Results}\label{Sec:Results}

\begin{itemize}

    \item Organize material and present results.

    \item Use tables, figures (but prefer visual presentation):
        \begin{itemize}
            \item Tables and figures should supplement (and not duplicate) the
                text.

            \item Tables and figures should be provided with
            legends.\\

            \item Tables and graphics may appear in the text or in
                the appendix, especially if there are many simulation results
                tabulated, but is also depends on the study and number of tables resp.
                figures. The key graphs and tables must appear in
                the text!
        \end{itemize}

    \item Latex is really good at rendering formulas:\\
        {\it Equation (\ref{Eq:SpecDens}) represents the ACs of a stationary
        stochastic process:
        \begin{equation}
            f_y(\lambda) = (2\pi)^{-1} \sum_{j=-\infty}^{\infty}
                           \gamma_j e^{-i\lambda j}
                         =(2\pi)^{-1}\left(\gamma_0 + 2 \sum_{j=1}^{\infty}
        \gamma_j \cos(\lambda j)\right)
                                        \label{Eq:SpecDens}
        \end{equation}
        where $i=\sqrt{-1}$ is the imaginary unit, $\lambda \in [-\pi,
        \pi]$ is the frequency and the $\gamma_j$ are the autocovariances
        of $y_t$.}

\newpage

    \item Discuss results:
        \begin{itemize}
            \item Do the results support or do they contradict economic theory ?
            \item What does the reader learn from the results?
            \item Try to give an intuition for your results.
            \item Provide robustness checks.
            \item Compare to previous research.
        \end{itemize}
\end{itemize}


This document was produced in RStudio using the knitr package \citep{Robson2009}.
\nocite{Freeman2005}
\citet{Freeman2005} is auch schön
 


% ----------------
% --- appendix ---
% ----------------
\appendix

% literature
\newpage
\addcontentsline{toc}{section}{Literatur}

\printbibliography


% figures (not mandatory)
\newpage
\section{Abbildungen}

hier kommen dann die Abbildungen hin


% tables (not mandatory)
\newpage
\section{Tabellen}

\begin{table}[h]
\centering
\begin{tabular}{@{}lrrrr@{}}
\toprule
                                       & \multicolumn{1}{r}{{\bf Gentri}} & \multicolumn{1}{r}{{\bf Kontroll}} & \multicolumn{1}{r}{{\bf Andere}} & \multicolumn{1}{r}{{\bf \textit{Gesamt}}} \\ \midrule
$m$                                    & 45                               & 139                                & 246                              & 430                                  \\
$Q_{0.5}(EW_{2007})$                   & 9734                             & 7172                               & 6309                             & 6907                                 \\
$\Sigma\;EW_{2007}$                    & 456355                           & 1158061                            & 1727819                          & 3342235                              \\ \bottomrule
\end{tabular}
\caption{Statistiken zur Einwohner*innenzahl nach \textit{Kategorie}}\label{tab:KategorieEW}
\end{table}            



% --------------------------------------------
% --- last page: Declaration of Authorship ---
% --------------------------------------------

\newpage
\thispagestyle{empty}
\input{Andere/Eigenstaendigkeitserklaerung}

\end{document}

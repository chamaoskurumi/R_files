\begin{center}
    {\normalsize{\bf Masterarbeit}} \\\vspace{0.5cm}
    {\LARGE{\bf Räumlich Autoregressive Modelle zur Messung von Verdrängung in Berlin}} \vspace{0.5cm}

    {\normalsize eingereicht beim Betreuer}\\\vspace{0.5cm}
    {\normalsize{\bf Prof. Dr. Axel Werwartz}} \\\vspace{0.5cm}
    {\normalsize Technische Universität Berlin \\
    Fakultät VII - Wirtschaft und Management \\
    Institut für Volkswirtschaftslehre und Wirtschaftsrecht \\
    FG Ökonometrie und Wirtschaftsstatistik} \vspace{1cm}

    {\normalsize sowie beim Zweitgutachter}\\\vspace{0.5cm}
    {\normalsize{\bf Dr. Andrej Holm}} \\\vspace{0.5cm}
    {\normalsize Humboldt-Universität zu Berlin \\
    Kultur-, Sozial- und Bildungswissenschaftliche Fakultät\\
    Institut für Sozialwissenschaften\\
    Stadt- und Regionalsoziologie} \vspace{1cm}

    {\normalsize vorgelegt von \\\vspace{0.5cm}
    {\bf Guido Schulz} \\
    Matrikelnummer: HU 554142 \\
    \href{mailto:schulzgu@student.hu-berlin.de}{schulzgu@student.hu-berlin.de}} \vspace{1cm}


    {\normalsize Zur Erlangung des akademischen Grades eines \\
    {\bf Master of Science (M.Sc.) in Statistik} \\
    Berlin, der 31. Juni 2015}

\end{center}
